To obtain a measure of the Fierz term, the beta decay spectrum must be precisely described.
The description of the beta decay spectrum is writen as a series of corrections on top of the main phase space factor.  

The corrections depend on several parameters of the decay. 
Some of them are just numbers, such as the Z of the daughter or mother nucleus or the A of the system.
However, there are two important paramters that are inexact.
They are measured experimentally.

One is the q-value of the decay, which is defined in equation \ref{eq:qval}.

\begin{equation}
	Q = m_{p} - (m_{d} + E_{\gamma})
	\label{eq:qval}
\end{equation} 
where $Q$ is the q-value, $m_{p}$ is the mass of the parent nucleus, $m_{d}$ is the mass of the daughter nucleus, and $E_{\gamma}$ is the energy of any gamma rays in the decay.
For this measurement, the atomic mass of $^{20}$Ne and $^{20}$F were used \cite{Pfe12}.
The mass of the electrons subtracts out atomic masses, since in neutral fluorine there are nine electrons, and one electron is gained from the beta decay.
This then equals the ten electrons in the atomic mass of neon.
The only correction missing is the small, eV scale electron binding energy of the last electron.

The other parameter is the charge radius of the daughter nucleus.
There are several ways to calculate the charge radius.
In this work, the charge radius was taken from the measured RMS charge radius and converted.
It was assumed that the $^{20}$Ne nucleus was a sphere. 
From this, the radius was calculated using equation \ref{eq:sphereeq}

\begin{equation}
	r = \sqrt{\frac{5}{3}}r_{rms}	
	\label{eq:sphereeq}
\end{equation}

where $r_{rms}$ is the root mean square radius, and $r$ the charge radius.

With the two parameters, the q-value and the phase space, the shape of all beta decay spectra can be described.

\section{Phase Space Beta Decay}
The main part of the beta decay spectrum is the phase space factor.
It is shown in equation \ref{eq:phase_space}

\begin{equation}
	\frac{dN}{dE} = C * p_{\beta}W(Q - W)^{2}
	\label{eq:phase_space}
\end{equation}

were $C$ is a constant, $p_{\beta}$ is the beta momentum, $W$ the total electron energy, and $Q$ the q-value of the beta decay.
This is derived from the density of states of the particle.
This comes in from Fermi's Golden Rule.

The number of states for both the electron and the neutrino is show in equation \ref{eq:densityofstates}

\begin{equation}
	N = \frac{1}{(2\pi\hbar)^{6}}\int dr^{3}_{\beta} \int dp^{3}_{\beta}\int dr^{3}_{\nu_{e}} \int dp^{3}_{\nu_{e}} 
	\label{eq:densityofstates}
\end{equation}

where $r$ and $p$ corresponds to the position and momentum of the electron ($\beta$) and the neutrino ($\nu_{e}$).
Doing the integral over both $r$'s give two factors of the volume, which disappear when the next factor is introduced.
Since the neutrinos are unmeasured, the momentum of the neutrinos is integrated over. 
Assuming spherical symmetry, the new integral is shown in equation \ref{eq:dosspherical}

\begin{equation}
	dN = \frac{V^{2}}{4\pi^{4}\hbar^{6}}p_{\beta}^{2}dp_{\beta}p_{\nu_{e}}^{2}dp_{\nu_{e}}
	\label{eq:dosspherical}
\end{equation}

This is simplfied by approximating the neutrino as massless.
The momentum of the neutrino is then equal to the energy of the neutrino.
Then, the total energy $E$ is written as a sum of the neutrino energy (or momemntum) and the beta energy.
Plugging that in gives equation \ref{eq:dosintegral} after integrating over the neutrino degrees of freedom. 

\begin{equation}
	dN = \frac{V^{2}}{4\pi^{4}\hbar^{6}}(E - E_{beta})^{2}p_{\beta}^{2}dp_{\beta}
	\label{eq:dosintegral}
\end{equation}

Then, the only thing left to do to recover equation \ref{eq:phase_space} is to rewrite $dp$ in terms of $dE$. 
The energy $E_{beta}$ has been replaced with $W$, which is the energy of the electron divided by the mass an electron, and $E$ has been replaced by the Q-value.
Since the measurement is not an absolute one, but only concerned with the shape of the spectrum, the normalization factor is arbitrary.

This is the main factor in describing the beta decay energy spectrum.


\section{Fermi Function}

The first correction is the Fermi function.
This accounts for the interaction of the charge of the outgoing electron and the charge of the nucleus.
 
