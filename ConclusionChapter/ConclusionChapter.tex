\documentclass[main.tex]{subfiles}

\begin{document}
The central value is not set in stone yet.
However, most of the systmatic effects can still be calculated.
The effect with the lower beta cut is likely due to the normalization of the beta-gamma sum spectrum.
The issue is a solvable oen. 

\section{Discussion of Results}

The statistical uncertainty of the lower beta cut effect is 0.0042.
This is with 7 million events that pass the filter.
If the lower beta cut uncertainty is excluded, the total systematic uncertainty is 0.0036.
This also excludes any uncertainty on the offset in the calibration and the normalization of the gamma-beta sum spectrum.
This is dominated by the uncertainty on the weak magnetism.
If the lower beta cut uncertainty can be resolved, the total systematic uncertainty should only be a touch higher than 0.0036.
This is very close to the statistical uncertainty. 

To lower the systematic uncertainty, the weak magnetism should be known to a more percsise value.
The uncertainty of the weak magnetism comes from how it is calcualted.
The parameters need to cacluated the weak magnestism are shown in equation \ref{eq:bwmcal} .
Most of the uncertainty comes from the gamma decay strength \cite{Min11}.
A better measurement of this gamma decay strength would reduce the effect of the weak magnetism.

\section{Results Compared to Other Measurements}

Another way of measuring the Fierz term in a Gamow-Teller transition is to look at neutron decay \cite{Hic17}.
This was done with ultra-cold neutrons, and the results are blah.
The beta decay energy spectrum is seen. 

\section{Further Refinements to Technique}
\end{document}
