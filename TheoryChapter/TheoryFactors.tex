
To obtain a measure of the Fierz term, the beta decay spectrum must be precisely described.
The description of the beta decay spectrum is writen as a series of corrections on top of the main phase space factor.  

The corrections depend on several parameters of the decay. 
Some of them are just numbers, such as the Z of the daughter or mother nucleus or the A of the system.
However, there are two important paramters that are inexact.
They are measured experimentally.

One is the q-value of the decay, which is defined in equation \ref{eq:qval}.

\begin{equation}
	Q = m_{p} - (m_{d} + E_{\gamma})
	\label{eq:qval}
\end{equation} 
where $Q$ is the q-value, $m_{p}$ is the mass of the parent nucleus, $m_{d}$ is the mass of the daughter nucleus, and $E_{\gamma}$ is the energy of any gamma rays in the decay.
For this measurement, the atomic mass of $^{20}$Ne and $^{20}$F were used \cite{Pfe12}.
The mass of the electrons subtracts out atomic masses, since in neutral fluorine there are nine electrons, and one electron is gained from the beta decay.
This then equals the ten electrons in the atomic mass of neon.
The only correction missing is the small, eV scale electron binding energy of the last electron.

The uncertainty in the energy of the gamma ray is much larger than the uncertainty of the binding energy.
The decay of $^{20}$F of interest decays to an excited state of the $^{20}$Ne.
The energy of the gamma ray was measured to 0.015 keV \cite{Til98}.

Using the numbers described above and equation \ref{eq:qval}, the q-value for $^{20}$F is 5.901928 (82).
The largest source of uncertainty is in the mass of $^{20}$F.


The other parameter is the charge radius of the daughter nucleus.
There are several ways to calculate the charge radius.
In this work, the charge radius was taken from the measured RMS charge radius and converted.
It was assumed that the $^{20}$Ne nucleus was a sphere. 
From this, the radius was calculated using equation \ref{eq:sphereeq}

\begin{equation}
	r = \sqrt{\frac{5}{3}}r_{rms}	
	\label{eq:sphereeq}
\end{equation}

where $r_{rms}$ is the root mean square radius, and $r$ the charge radius.

With the two parameters, the q-value and the phase space, the shape of all beta decay spectra can be described.

\section{Phase Space Beta Decay}
The main part of the beta decay spectrum is the phase space factor.
It is shown in equation \ref{eq:phase_space}

\begin{equation}
	p_{e}W(Q - W)^{2}
	\label{eq:phase_space}
\end{equation}

were $p_{e}$ is the electron momentum, $W$ the total electron energy, and $Q$ the q-value of the beta decay.
This is derived using Fermi's golden rule. 
