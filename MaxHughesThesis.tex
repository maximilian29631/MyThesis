%% Before beginning to type your dissertation, download and READ 
%% the Graduate School's formatting guide, which can be found at 
%% http://grad.msu.edu/etd
%% and clicking Formatting Guide in the left hand column.
%% Also get the latest version of  msuphddissertation.cls and the template file
%% at http://www.math.msu.edu/~weil/MSU_Ph.D._Dissertation.zip
%% Send questions to weil@math.msu.edu

%%%%%%%%%%%%%%%%%%%%%%%%%%%%
%%%%%%%%  NOTE   %%%%%%%%%%%%%%
%% PREPARING A DISSERTATION WITH THIS CLASS FILE DOES NOT %%%
%% GUARANTEE THAT THE GRADUATE SCHOOL WILL APPROVE IT. %%%
%%%%%%%%%%%%%%%%%%%%%%%%%%%%%%%

%% To view a video presentation of this template, visit
%%  https://www.math.msu.edu/latex/dissertation/

\documentclass{msuphddissertation}

\usepackage{subfiles}
\usepackage{graphicx}
\usepackage{xr}
\usepackage{amsmath}

\graphicspath{ {Figures/}}


%% This is the first command that must appear in your thesis.
%% Insert packages you wish to use except setspace, subfig
%% geometry and pdflscape. 
%% These packages are loaded automatically.
%% IMPORTANT: Load only those packages you know you will use.
%% Some packages can cause conflicts resulting in improper formatting.
\author{Maximilian Nathan Hughes} %% Put your name in full as it is officially recognized by Michigan State University here.
\title{Precision Measurements of $^{20}$F Beta Decay} %% Put the title of your dissertation here.
%% Go to http://grad.msu.edu/etd/docs/DegreeGrantingUnits.pdf
%% and find your GRADUATE DEGREE GRANTING UNIT/PROGRAM
%% and DEGREE.
\unit{} %% Copy and paste these two items here 
%% separated by a dash, created by typing ---.
%% ONLY ENTRIES FROM LIST MENTIONED ARE ACCEPTABLE!!

%% Put additional preamble items here.

\begin{document}

\maketitlepage %%This command will produce the title page of your thesis.

%% If you wish to include a "public abstract" (i.e.; in layman's terms), remove the "%" 
%% in from of the command \begin{pub abstract} and remove the "%" in front of
%% \end{pub abstract} below. A public abstract isn't required, but might be useful
%% for some readers.
%\begin{pubabstract}
%%Type the text of your public abstract here. A public abstract is optional.
%\end{pubabstract}

\begin{abstract}
%% Type your abstract here. An abstract is REQUIRED and limited to two pages.
%% The abstract must not include any figures.
\end{abstract}


%% If you wish to have a copyright page, remove the "%" 
%% in front of \begin{copyrt}
%% and remove the "%" in front of \end{copyrt}.
%% An acceptable form of a copyright page  
%% will be generated automatically. 
%% TO INCLUDE A COPYRIGHT, YOU MUST REGISTER
%% IT. See the Formatting Guide for instructions. 
%\begin{copyrt}
%\end{copyrt}


%% If you wish to have a dedication, remove the "%" in front of
%% \begin{dedication} and remove the "%" in front of
%% \end{dedication} below.
%% A dedication must be single-spaced and 
%% centered on the page. Both will be done automatically. 

\begin{dedication} 
%% Type your dedication here. A dedication is optional.
\end{dedication}

%% If you wish to have an acknowledgment, remove the "%" in front of  \begin{acknowledgment}
%% and remove the "%" in front of  \end{acknowledgment} below.  
\begin{acknowledgment}
%% Type your acknowledgment here. An acknowledgment is optional.
\end{acknowledgment}

%% If you wish to have a preface, remove the "%" in front of \begin{preface}
%% and remove the "%" in front of \end{preface} below. The formatting of
%% a preface isn't specified, but it is included in the TOC.
%\begin{preface}
%% Type your preface here. A preface is optional.
%\end{preface}

\TOC %% This command produces the Table of Contents. DO NOT REMOVE IT!

%% If your document contains tables, remove the "%" in front of 
%%  the following line.
\LOT

%% If your document contains figures, remove the "%" in front of
%% the following line.
\LOF

%%%% LIST OF SYMBOLS OR LIST OF ABBREVIATIONS %%%%
%% If you wish to have a list of symbols or a list of abbreviations, 
%% it should be here. For a list of symbols remove the "%" in front of 
%% \begin{symbols} and remove the "%" in front of \end{symbols} below.
%\begin{symbols}
%% Type your list using a list environment here.
%\end{symbols}
%% Similarly for a list of abbreviations remove the "%" in front of 
%% \begin{abbrev} and remove the "%" in front of \end{abbrev} below.
%\begin{abbrev}
%% Type your list using a list environment here.
%\end{abbrev}
%% The list will be included in the TOC as
%% KEY TO SYMBOLS or KEY TO ABBREVIATIONS
%%%%%%%%%%

\newpage
\pagenumbering{arabic}
\begin{doublespace}

\chapter{Introduction}
\subfile{IntroChapter/Introduction.tex}

\chapter{Fierz Term and $^{20}$F}
\subfile{FierzTermand20FChapter/FierzTermand20F.tex}

\chapter{Theoretical Description of Beta Decay Spectrum}
\label{ch:theory}
\subfile{TheoryChapter/TheoryFactors.tex}

\chapter{Experimental Description}
\subfile{ExperimentDesChapter/ExpDiscription.tex}

\chapter{Data Processing}
\subfile{DataProcessing/DataProcessChapter.tex}

\chapter{Half Life Measurement}
\subfile{HalfLifeChapter/HalfLifeSection.tex}

\chapter{GEANT4 Monte Carlo}
\subfile{Geant4Chapter/Geant4Chapter.tex}

\chapter{Fitting Beta Spectrum}
\subfile{BetaFitChapter/BetaFitChapter.tex}

\end{doublespace}

%% Put the body of your dissertation here. 
%% DO NOT include the bibliography or any appendices.
%% These topics will be discussed later.

%%%%%% LANDSCAPE PAGES  %%%%%%
%% To produce graphics or tables in landscape mode,
%% begin by removing the "%" in the next two lines.
%\begin{landscape}
%\thispagestyle{empty}
%% The contents of the page can be centered using the center environment
%% or the \centering command. Insert either a table with the tabular environment
%% or input a graphics file.
%% Use \captionof{table}{caption_text} (or figure in place of table) 
%% to create the caption for a short table or a figure. 
%% Insert a long table in a table environment. It disables double spacing
%% which is permitted for long tables. Finally remove the "%" from the next line.
%\end{landacape}

%%%%%%%    APPENDICES    %%%%%%%%%%
%% If you wish to include just one appendix, remove the "%" 
%% in front of \appendix below. To include two or more appendices,
%% remove the "%" in front of \appendices.
%\appendix
%\appendices

%% In either cast to start your first appendix, which will be labeled
%% as Appendix A, just type \chapter{<appendix 1 name>}
%% and enter the text of the appendix as you would a chapter,
%% with one exception. If you use any subdivisions, such as 
%% chapter, subchapter etc., use the starred version; that is,
%% \chapter*{chapter name}.  Such subdivisions are not to be 
%% listed in the Table of Contents.

%%%%%%% A NOTE ABOUT APPENDICES %%%%%%%%%
%% Some appendices may be single-spaced such as survey examples
%% or letters. See the Graduate School's formatting guide for details.
%% To single space an appendix first remove the "%" from 
%% the following two lines.
%\end{doublespace}
%\chapter{<appendix name>}
%% Insert the name of the appendix.
%% Insert the text of the appendix.
%% Remove the "%" from the following line.
%\begin{doublespace}
%% Any text entered now will be double spaced.


%%%%%%  Bibliography %%%%%
%% A bibliography is required. By default it is called, "Bibliography"
%% You may use �LITERATURE CITED�, �WORKS CITED� or �REFERENCES� 
%% instead of �BIBLIOGRAPHY� if that is the convention in your discipline. 
%% To do so, copy and paste your choice into the empty argument 
%% of the following command and remove the "%".
%\renewcommand{\bibname}{}
\bibliography{./Biliography/thebibliography}
\bibliographystyle{plain}
%% The bibliography may be made using BibTeX.
%% To do so the necessary commands must be entered in the 
%% preamble and here.
%% If the Bibliography is made from scratch,
%% remove the "%" in front of \begin{thebibliography}{???}
%% replacing the ??? with the appropriate entry and 
%% remove the "%" in front of \end{thebibliography} below.
% \begin{thebibliography}{???}
%%  Enter the bibliography here.
% \end{thebibliography}
%% In either case, the bibliography is automatically entered
%% in the Table of Contents.
\end{document}

%%%%%% FINAL COMMENTS %%%%
%% Before submitting your dissertation to the Graduate School
%% Make sure there isn't any text in the right margins. To do 
%% in the .log file look for error messages beginning with, 
%% "Overfull \hbox ". 

%% Once your document has been filed with the Graduate School,
%% if you wish to produce a single spaced version of your document, 
%% find and remove the commands \begin{doublespace}
%% and \end{doublespace} above.
