\externaldocument[I-]{MaxHughesThesis}

In order to get a measurement of the Fierz term, the spectrum shape needed to get fit.
The effect of bremstrahlung and the effieciency of the detectors needed to be accounted for.
The way this was done was with a a simulation.

\section{GEANT4 Monte Carlo}
The corrected beta decay spectrum was fed as input to a Monte Carlo.
The program used to model the detectors was GEANT4.
In order to use the program, parameters must be introduced.

\subsection{Detector Simulation}
The geometry of the detector set-up was programmed into the simulation.
The implant detector was modeled as a square prism of CsI.
The rectangular prism was 9.76 cm deep with a 5 cm square base.
The front edge implant detector was put at the center of the simulation.
The aluminum sheath and MgO layer was not simulated for the implant detector.

The four large gamma detectors also square prisms.
The active volume was 79.5 mm square and 76.2 mm deep.
It was also made of CsI.
There was a 2 mm dead layer of vacuum around the detector.
Above this dead layer, a 1.5 mm laywer of MgO was added.
On top of this, the can of aluminum, 1 mm thick, was added into the simulation.

The four large gamma detectors were arranged into a square around the implant detector.
Each square base of the gamma detectors was centered one inch upstream from the face of the implant detector.
This modeled how the implant detector was recessed in the experiment.

\subsection{Source Definition}
The next step was to define a region inside the implant detector.
This region was where the gamma and beta particles orignated from.
The depth of the region was calculated using LISE++, a ion optics code.
The verticle and horizontal size of the region was calulated by using the PPAC measurement and an ion optics simulation.
This size was 0.4 mm deep, 3.5 mm wide, and 3.6 mm tall.
The source was implanted 1.156 cm into the detector.

Once these parameters were known, GEANT4 was told to generate two particles.
First, the corrected beta energy spectrum was sampled and an electron of that energy was generated inside the implant region.
The beta energy spectrum was generated using all the corrections described in chapter \ref{ch:theory}.
The location of the electron was recovered, and a photon with an energy of  1.6 MeV was generated as well.
The initital direction of the two particles was as isometric.
These particles where then propagated through the detector setup..
GEANT4 uses a physics list that has all the interactions a particle could take.
The physics list used was standard option 4.
This list is better at low energy electron physics.

\subsection{Primary Particle Definitions}
There were three primary particles generated.
Two were photons and one was an electron.
All three particles had an isotropic angular distribution.
The origin points of all the primary particles was the same.

One photon was the 1.6336 MeV photon from the $^{20}$F decay.
The energy of this photon was not changed.

The electron was generated according to phase space spectrum (equation \ref{eq:phase_space}) times all the corrections.
For the radiative correction, the electron was generated with the formala that assumes all real photon energy is absorbed (equation \ref{eq:fayansrad}).
However, this is not exactly true. 
In this case, due to the geometry, photons are absorbed totaly if they are below 100 keV in energy.
In order to properly see the effect of the real photon emmission, the second primary particle is used.

After the electron energy is generated, a formula describing the energy spectrum of the inner bremsstrahlung photons is generated. 
This spectrum is written out in equation \ref{eq:KUB}. % Add KUB again here? Probably.
Further discussion of this formula is found in the theory chapter.
A cutoff of 50 keV is imposed to the formula, as it has a singularity at zero photon energy.
Then, the formula is numerical integrated from 50 keV to the electron energy.
This is the total probablity that an electron emits a KUB photon.
This number is compared to a random number from 0 to 1.
If the random number is below the integral, the inner bremsstrahlung energy spectrum is sampled.
The algorithm used for this sampling is the van Neumann method \cite{neu51}.
The sampled energy is given to the third primary particle.
The energy of the electron is reduced accordingly.

After all three particles have their energies defined, they are propogated through the detector set up.

\section{MC Output}
The particles were tracked and the energy deposited in each detector summed up.
The energy deposited in the implant detector was further seperated into two parts:
Energy from the  electron and the inner bremstrahlung photon.
Energy from the 1.6336 MeV photon
After the energies of the particles reached a certain threshold, the simulation of one decay was finished.
All the energies deposited into each of the detectors was summed up and saved as an event in a ROOT tree.
Then, the process was repeated.
A new location inside the region was generated and another decay generated.

In order to get to get the necessary statistics, 2 * 10$^{9}$ events had to be generated. 


