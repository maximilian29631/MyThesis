\documentclass[../MaxHughesThesis.tex]{subfiles}

\begin{document}
Due to uncertainties in how to deal with the gamma-beta sum spectrum, the central value of the Fierz term is not ready to be stated. 
However, most of the systematic effects can still be calculated.
The effect with the lower beta cut is likely due to the normalization of the beta-gamma sum spectrum.
The issue is a solvable one. 

\section{Discussion of Results}

The statistical uncertainty of $b_{GT}$ is 0.0042.
This is with 7000000 events that pass the gamma coincidence.
If the lower beta cut uncertainty is excluded, the total systematic uncertainty is 0.0066.
This also excludes any uncertainty on the normalization of the gamma-beta sum spectrum.
This is dominated by the uncertainty of the offset.
If the lower beta cut uncertainty can be resolved, the total systematic uncertainty should only be a touch higher than 0.0066.
This is larger than the statistical uncertainty.
The largest systematic uncertainty is due to the offset, however.
Increasing the number of events should decrease this uncertainty.
The next largest systematic effect is the uncertainty on the $b_{WM}$.
Increasing the number of counts will not decrease the uncertainty of $b_{GT}$ below 0.0028.
A new measurement of $b_{GT}$ would need to be done.

%In order to compare this to high energy techniques, equation \ref{eq:bgtpropor} is used \cite{Gon19}.
%If the lower beta cut effect is resolved and the systematic effects described above accurate, then the total uncertainty on $b_{GT}$ is 0.0078. 
%This corresponds to an uncertainty on $Re(\epsilon_{t})$ of 0.0013.
%This is about twice the uncertainty from the neutral current channel at the LHC. 
%Running the experiment with more statistics would help reach that uncertainty.
%Then, the statistical uncertainty would be negligible, and the systematic uncertainty would be the limiting factor. 

\section{Further Refinements to Technique}
The first thing to do to improve the experiment is to run more events.
Most of the events ran for this experiment were with the PVT detector.
However, the shape of beta spectrum in that detector had an odd distortion in it.
If that could be investigated and solved, using a plastic scintillator would be an ideal candidate for this measurement.
Less of the gamma ray would be absorbed, and there would be less bremsstrahlung in the detector.

The largest systematic effect right now is the lower beta cut effect. 
A likely way to solve this issue is to adjust the level of the gamma-beta sum spectrum.
This still to be confirmed with Monte Carlo.
Ultimately, the resulting offset and sum spectrum level probably are not the real, physical level.
An estimate of the systematic uncertainty due to the sum spectrum level would then have to be done, much like with the offset.
As these uncertainties depend on the slopes of the lower beta cut curves, these should get more precise as more events are built.
The slope can be seen more easily as there is the error bars get smaller.
Another method would be to compare the output of GEANT4 to that of another simulation code.
EGSnrc is being used by the collaborators from Wittenburg University. 
These results will be added to the systematic uncertainty.

In order to increase efficiency for the 1.6 MeV gamma ray, the implant detector could be more completely encased with gamma detectors.
If these gamma detectors had a higher resolution, that would mean that the signal in the implant would be much cleaner.
If that were coupled with a plastic detector, then the beta spectrum of the implant detector would be much cleaner.
This would hopefully eliminate the amount of gamma-beta sum spectrum and reduce the amount of bremstrahlung.

If these systematic effects are reduced, the next largest one has to do with the uncertainty on the measurement of $b_{WM}$.
To decrease the uncertainty of $b_{WM}$,  a new and better measurement of the width of the isobaric analogue state in $^{20}$Ne would have to be measured.
This is the parameter $\Gamma_{M1}$ in equation \ref{eq:bwmcal}.
This is the parameter that dominates the uncertainty in  the calculation of $b_{WM}$ \cite{Min11}.
This would require a different type of experiment.

\section{Conclusion}
With all this in place, this measurement of the Fierz term can be one of the more precise ones, if the issue with the lower beta cut can be figured out.
Many of the systematic effects would be present in other such measurements.
The caloremetric technique could be used for other allowed Gamow-Teller transitions.
Those with higher maximum energies would need an accelerator powerful enough to implant the beam deep enough to absorb all the electron energy.
A nucleus with a lower maximum energy, such as $^{6}$He, would be more sensitive to the 1/E dependence of the Fierz term. 
This technique is a promising method for precision measurements in nuclear beta decay.

\end{document}
