\documentclass[../MaxHughesThesis.tex]{subfiles}

\begin{document}
The half-life in $^{20}$F has been measured to be  $1.0011 \pm   0.0069_{\rm{stat}}  \pm 0.0030_{\rm{sys}}$ s.
This result has been published and is the most precise measurement of the half-life of $^{20}$F to date.
A preliminary result of the Fierz term has been obtained and is $0.0021 \pm 0.0051_{\rm{stat}} \pm 0.0084_{\rm{sys}}$.
Using equation \ref{eq:bgtpropor}, a constraint on the underlying $Re(\epsilon_{T})$ is that $Re(\epsilon_{T}) = 0.00034 \pm 0.00082_{\rm{stat}} \pm 0.00136_{\rm{sys}}$.
This assumes that the only new physics beyond the standard model is the tensor coupling. 
The constraint on $Re(\epsilon_{T})$ is comparable to the constraint obtained with high energy methods.

\section{Outlook}
This analysis is not yet complete.
To complete the analysis, first several systematic effects have to be added and reevaluated.
The uncertainty on the normalization of the beta-gamma sum spectrum can be re-evaluated.
The energy deposited in the gamma can be similarly split into two parts.
Then, by fitting the gamma spectra with the relative normalization as a free parameter, the uncertainty can be determined more accurately.
This will likely reduce the uncertainty due to this effect.
Another method would be to compare the output of GEANT4 to that of another simulation code.
EGSnrc is being used by the collaborators from Wittenburg University. 

Two systematic effects still have to be evaluated.
Both deal with the inputs of the GEANT4 Monte Carlo.
The first is the uncertainty due to different physics lists. 
This will require running GEANT4 again.
The second uncertainty has to do with an uncertainty due to the detector geometry.
The detector geometry will have to be varied by some amount, and GEANT4 run again.

The fitting method needs revision.
Both $b_{WM}$ and $b_{GT}$ are correlated.
To account for the correlation, a constraint fit would be needed.
The method of this fit is to fix both $b_{WM}$ and $b_{GT}$ at some reasonable value and fit for a common gain.
The total $\chi^{2}$ of the fit is recorded, and a new set of parameters fit. 
The parameters would be varied over a region, and the $\chi^{2}$s plotted.
The minimum of the $\chi^{2}$ would be located, and contours around the minimum would be drawn. 
The contours would correspond to the 1-$\sigma$, 2-$\sigma$, and 3-$\sigma$ uncertainties of the parameters. 
The projection of the 1-$\sigma$ contour on each axis would give the uncertainties for each parameter.
The systematic uncertainties would have to be recalculated.

\subsection{Technique Improvements}
For future experiments, other improvements of this technique could be done.

Ideally, the lowest lower beta cut should be used for the fit.
This both increases the statistics in the spectrum and decreases the uncertainty due to the uncertainty in $b_{WM}$.
Going below $200$ keV for this measurement is not possible due to the non-linearities in the detector.
If the issue with the shape of the PVT detector spectrum could be solved, the spectrum could be measured down to smaller energies, since plastic detectors are linear down to about 125 keV \cite{Kno10}. 
There would be an increased sensitivity to the Fierz term due to the $1/E$ functional dependence of that term.

In order to increase the efficiency for the 1.6 MeV gamma ray, the implant detector could be more completely encased with gamma detectors.
If these gamma detectors had a higher resolution, that would mean that the signal in the implant would be much cleaner.
If that were coupled with a plastic detector, then the beta spectrum of the implant detector would be much cleaner.
This would hopefully eliminate the amount of gamma-beta sum spectrum and reduce the amount of bremsstrahlung.

If these systematic effects are reduced, the next largest one has to do with the uncertainty on the measurement of $b_{WM}$.
To decrease the uncertainty of $b_{WM}$,  a new and better measurement of the width of the isobaric analogue state in $^{20}$Ne would have to be measured.
This is the parameter $\Gamma_{M1}$ in equation \ref{eq:bwmcal}.
This is the parameter that dominates the uncertainty in  the calculation of $b_{WM}$ \cite{Min11}.
This would require a different type of experiment.

\section{Conclusion}
This calorimetric technique could be used for other allowed Gamow-Teller transitions.
These other transitions would need to have a more precisely know $b_{WM}$ to a more precise value.
Those with higher maximum energies would need an accelerator powerful enough to implant the beam deep enough to absorb all the electron energy.
A nucleus with a lower maximum energy, such as $^{6}$He, would be more sensitive to the 1/$E$ dependence of the Fierz term. 
This technique is a promising method for precision measurements in nuclear beta decay.

\end{document}
