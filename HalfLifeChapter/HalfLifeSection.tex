%%%%%%%%%%%%%%%%%%%%%%%%%%%%%%%%%%%%%%%%%%%%%%%%%%%%%%%%%%%%%%%%%%%%%%%%%%%%
%\documentclass[a4paper,prc,unsortedaddress,superscriptaddress,showpacs,twocolumn,floatfix]{revtex4}

%\pacs{21.10.-k, 21.10.Tg, 27.20.+n} 
%%%%%%%%%%%%%%%%%%%%%%%%%%%%%%%%%%%%%%%%%%%%%%%%%%%%%%%%%%%%%%%%%%%%%%%%%%%%
%\maketitle
%\section{Experimental setup}
%\label{sec:exp}

%%%%%%%%%%%%%%%%%%%%%%%%%%%%%%%%%%%%%%%%%%%%%%%%%%%%%%%%%%%%%%%%%%%%%%%%%%%%
\section{Motivation}
During the experiment, it was noticed that there was a disagreement between different values of the half-life.
As a side project and as a chance to thoroughly investigate the data, a half-life measurement was taken. 
It has been used in the past for searches of second-class currents.


\section{Half life data analysis}
\label{sec:analysis}


\subsection{Data Selection}
Only the data from the PVT impantation set was used for the half-life analysis.
The rate depedent gain effect causes a larges systematic effect in the CsI(Na) data.
This effect causes the gain at early times is higher than the gain at later times in the decay cycle.
The cuts, however, stay in the same position.
Early in the decay cycle, events near the cuts could start out inside the window.
As time goes on and the gain changes, these cuts could move below the lower beta cut. 
Due to the shape of the beta spectrum, the higher the lower beta cut, the larger the effect.
This effect is seen in Figure \ref{fig:LBCvHL}.

\begin{figure}[!htb]
	\centerline{\includegraphics[width=0.78\textwidth]{CsILBCvHL.png}}
	\caption{The effect of the lower beta cut vs the half-life}
	
	\label{fig:LBCvHL}
\end{figure}

Attempts were made to counteract this effect with a more sophisticated analysis.
The beta spectrum was built and fitted second by second in order to obtain a calibration.
The gain of the calibration was plotted as a function of time and fit with a line.
The offest of the calibration was seen to be independent of time.
The resulting time-dependent calibration was used to analyze the data.
This reduced the effect seen in Figure \ref{fig:LBCvHL}, but did not entirely eliminate it.
Since the CsI(Na) data has a small statistical impact, it was decided to elliminate that data from the analysis of the half-life.
The cause of the rate dependent gain effect is afterglow in the scintillator.
The plastic scintllator does not have any afterglow, so the PVT implant detector is immune to this effect. 

The PVT data was seperated into 6 sets to check systematics.
The splitting was done by the rates and the gains of the different detectors.
The first and second sets did not have the limiter box installed.
This was to test the effect of the dead time correction across different rates.
Other systematic effect differences were investigated.

\subsection{Cut Selection}
In order to do the half-life analysis, software coincidences were imposed.
First, events from the energies and times of each detector were built.
Two conditions were on the energy in the implantation detector and one of the four CsI(Na) detectors.
An additional condition on the time difference between the events recorded in the two detectors. 
A sample spectrum of all the gamma and beta energies is shown in Figure \ref{fig:2DGraph}  

\begin{figure}[!htb]
	\centerline{\includegraphics[width=0.78\textwidth]{fig_2Dh.pdf}}
	\caption{The 511 region can be seen in this graph}
	\label{fig:2DGraph}
\end{figure}

\begin{figure}[!htb]
	\centerline{\includegraphics[width=0.78\textwidth]{fig_betaSpec.pdf}}
	\caption{The lower beta cut was selected to be above the 511 region.
		 The upper beta cut was selected to include all the pile up}
	
	\label{fig:BetaGraph}
\end{figure}

\begin{figure}[!htb]
	\centerline{\includegraphics[width=0.78\textwidth]{fig_gammaSpec.pdf}}
	\caption{The wide cuts are to reduce the effect of the rate dependent gain.}
	\label{fig:Gammaraph}
\end{figure}

Hard coded gates were used for to select in the beta window for the PVT run sets.
Each different data set had a different energy window due to the gain shifts.
A sample spectrum with the gates can be seen in Figure \ref{fig:BetaGraph}.
This spectrum is put in coincidence with the gamma spectra. 
The lower beta cut was selected to cut out the 511 beta spectram as seen in Figure \ref{fig:2DGraph}.
When gated around that energies, the resulting gammas measured by the large CsI(Na) detectors are consistent with $^{10}$C and $^{11}$C.
This comes from the break-up of the plastic. 

The set of runs before the limiter box was installed have a gain shift within each run, similar to the rate dependent gain effect.
This causes the half-life to vary greatly with the beta cut.
As the beta gates are moved, different counts move across it over the time interval, causing a change in lifetime. 
Only the runs after the installation of the limiter box were used for the half-life analysis. 
This excluded the first and second sets of data.

A time difference condition was set by building a spectrum of the time differences between the implant detector's event and one of the four gamma detector's events. 
The peak of this time difference spectrum was found, and an interval of $\pm$24 ns  was use for this gate.
A sample time difference spectrum can be seen in Figure \ref{fig:timediff}

\begin{figure}
	\centerline{\includegraphics[width = 0.88\textwidth]{fig_tbgSpec1.pdf}}	
	\caption{This is a gamma and beta energy filted spectrum of the time difference between the up detector and the PVT implant detector.
		The tail on the right side of the large peak is due to pile up.
		The narrow time cuts are shown with a dotted line.
		The wide time cuts are shown with a solid line.}
	\label{fig:timediff}
\end{figure}


The time signals for the outer four gamma detectors were used for the half-life analysis.
This gave 4 independent measurements of the half-life per run. 
Only full decay cycles were used. 
This was done by finding the last beam on and putting a condition on the run time of the events.
The decays where the up detector stopped counting were ignored, as there were few enough that they did not make a difference. 

The dead time was corrected using measured rates and a dead time of 464 ns for the implant detector and 656 ns for the gamma detectors.  
The dead time was measured by building a spectrum of the time difference between consecutive time stamps.
The lowest time difference was taken as the dead  time. 
This was checked using the light pulser in the PVT detector.
For each gamma detector, a histogram of the energy-filtered rate was built.
The unfiltered implantation detector rate and gamma rate was built.
Using the uncorrected rates as an input, the gamma detector rate was corrected for bin by bin.
The correction used is shown in equation \ref{eq:dtc}
%
\begin{equation}
r^{c}_{coincidence} = \frac{1}{1 - r_{\beta}\tau_{\beta}}\frac{1}{1 - r_{\gamma}\tau_{\gamma}}r^{m}_{coincidence} 
	\label{eq:dtc} 
\end{equation}
%

where $r^{c}_{coincidence}$ is the corrected gamma-beta coincidence rate, $r_{\beta}$ the raw measured implant rate, $\tau_{\beta}$ the dead time of the implant detector, $r_{\gamma}$ and $\tau_{\gamma}$ the raw measured rate and dead time for the CsI(Na) detector, and $r^{m}_{coincidence}$ the measured coincidence rate.   
Then, after corrected for the dead time, each cycle was added together relative to the last beam off.
These stacked cycles were used to find the half life.

Once the decay spectra were put in coincidence and stacked up, it was fit with equation \ref{eq:fit-function}

%
\begin{equation}
	f(t) = a\exp{(-t*ln2/T_{\frac{1}{2}})}
	\label{eq:fit-function}
\end{equation}
%
where $a$ and $T_{\frac{1}{2}}$ are free parameters.
$a$ is the initial rate and $T_{\frac{1}{2}}$ the half-life.
The decay curves were fit from 1.5 seconds past the beam on time to 1.5 second from the end of the decay time. 
The fitting method used was the log likelyhood method. 
The 60 second decay run up detector result is shown in Figure \ref{fig:60secdecay}.
From this run, it is seen that the spectra does not decay back down to the background. 
The lack of background parameter in equation \ref{eq:fit-function} comes from this observation. 

\begin{figure}[!htb]
\centerline{\includegraphics[width=0.88\textwidth]{fig_decaySpec.pdf}}
\caption{The decay spectrum from the up gamma detector is shown on the top graph.
	The red line is the exponential fit. 
	The bottom graph shows the residuals from the fit. 
	}
\label{fig:60secdecay}
\end{figure}


The decay spectra were built run by run, and the resulting fitting results were averaged together. 
Runs with a p-value less than 0.05 were considered statistically insignificant and thrown out.
After all the significant half-lives were collected, the average of them all was taken.

\section{Systematic Effects}
Several systematic effects were looked at.
The dead time correction was applied before any other systematics were investigated. 

\subsection{Dead Time}
The timing resultion of the clock is 8 ns.
Due to this, there is an uncertainty on the dead times of at least 8 ns.
The dead times were varied $\pm$4 ns and the half lives calcluated.
Half the difference of those half lives is the systematic uncertainty.

\subsection{Pile Up}
The dead time is overcorrected. 
Due to pile up, some of the pile up events are not totally lost.
Some of the events thought to be missing just got shuffled around.
This is a problem due to the fact that energy gates are imposed.
To estimate the size of this effect the results from two different time cuts were compared.
The 48 ns time cut around the peak, shown in Figure \ref{fig:timediff}, was compared to a 300 ns time window.
There is still some pile up events under the prompt peak.
The number is one fifth of the number of pile up events in the wide cuts.
One fifth of the time difference between the wide cuts and the narrow cuts is the correction and systematic error.

\subsection{Background}

The first systematic effect is the effect of background.
Due to the low decay time, directly fitting a background is impossible.
There is no background region for the fitting function to anachor to, which induces a large correlation between the background and the half life.
Several techniques were used to estimate the background.

\subsubsection{Simultaneous vs Seperate Fitting}
The first attempt was by simultaneously fitting each run with a different fitting function.
The equation used was equation \ref{eq:fit-bg}. 
%
\begin{equation}
	f(t) = a(\exp{-t*ln2/T_{\frac{1}{2}}} + b)
	\label{eq:fit-bg}
\end{equation}
%

with $a$ and $T_{\frac{1}{2}}$ the same as in equation \ref{eq:fit-function} and $b$ the relative background level.
The half-life of the four detectors was set to be a common parameter. 
The relative background level and initial rate, however, were assumed to be independent for each spectrum.
The resulting half life was compared to the value found with fitting each detector seperately without a background.

The effect of adding a background was investigated using Monte Carlo.
Exponentials were generated and an ammount of background added.
The resulting function was fit with a decay curve without a background parameter.
The amount of background added was varried, and the resulting half lives plotted.
It was found that increasing the background decreased the value of the half life, and that the background level and the half life value were strongly correlated.

A similar Monte Carlo was written to check the simultaneous fit method.
Four decay curves were fitting and simultaneously fitted the same way was the data.
They were also fitted seperately with only an exponential decay.
The background was varied and the difference between the two values was taken.
The method shows a trend similar to the one seen in the previous Monte Carlo.

In the data, it was discovered that the results of the simultaneous vs seperate fitting depended on the size of the dead time correction.
If the dead time was overcorrected, it induced an effect that was similar to having a larger background.
If more dead time was imposed, the effect was as if there was a negative background.
A dead time correction was added to the Monte Carlo, and a similar trend was seen.
As long as the dead time is known exactly, the background can be extracted with this technique. 
However, as it is somewhat uncertain of what the exact dead time should be, this technique cannot be used to determine the background.

\subsubsection{Spectra Arguments}
The other way to try and gauge the background size is to look at the spectra.
Looking at Figure \ref{fig:timediff}, it can be seen that on the left side of the large peak, there are very few counts.
It can also be seen that the prompt peak in the center contains most of the counts, while there is a long tail on the right side of the peak.
This tail is due to pile up. 
An event in the beta window piles up with another event later in time. 
The second event is read by a gamma detector, but is correlated with the first event, causing the large time difference.
The time spectrum changes if the beta spectrum is cut in different ways.

In the gamma beta spectrum, it can be seen that there is no background events aside from the 511 region.
This is due to the presence of $^{10}$C and $^{11}$C.
When the detectors are put into tripple and quadruple coincidence, the gamma rays of those isotopes appear.
There are no gamma rays in the 1620 keV window which was filtered on..
The only coincidence possible is if there is pile up into that window along with a beta event.
The estimate of those events is shown on the left side of the peak in the time difference figure shown in Fig. \ref{fig:timediff}  
These accidental coincidence seem to suggest that the background level is negligible.


\subsection{Cut Sensitivity}
For each pair of detectors, there were four conditions: two for each the implant and gamma detectors. 
For the gamma cuts, the edges of the gates were varied $\pm$ 5 keV in each direction for each cut.
The results is insenitive to the upper beta cut.
The lower beta cut was scanned with 6 different values going evenly from the initial lower beta cut to the peak of the spectrum.
Moving the lower beta cut also effects the dead time correction.
In order to disentangle the two, the lower beta cut was varied with two different time difference conditions.
The location where the difference between the two calculations blows up is where the effect of the lower beta cut starts being swamped by the effect of the pile up.
This is limit of where the lower beta cut was scanned.
All four of these conditions were varied independently, and the procedure of generation the spectrum and fitting the decay curves was done.
Half the resulting range of half-life values was taken to be a systematic uncertainty.
	
The fitting range of the decay spectra was varied. 
The start of the fit was varied bin by bin up to 6.5 seconds into the decay spectrum.
The end of the fit was varied the other way to cut out the end of the decay spectrum.
Since there was no noticable systematic effect, there was no increase in error associated with this effect.

\subsection{Oscillator Stability}
The oscillator of the PIXE board has a stability of $\pm 5 * 10^{-5}$.
All times in the analysis were stretched by this value, and the half life calculated.
Have the difference between the stretched value and the original value is another systematic error.

\subsection{Binning and Fitting}
In order to check the senstivity of the result to binning, the decay curves were rebinned by a factor of two.
Half the difference between the rebinned half life and the original half life is considered the correction and the uncertainty.

For the fitting method, log-likelihood estimators were used to fit the summed data.
This was done with two different packages which gave identical results. 
This was compared to analytic results, which gave the same results, so the half life is insensitive to the fitting method.

%%%%%%%%%%%%%%%%%%%%%%%%%%%%%%%%%%%%%%%%%%%%%%%%%%%%%%%%%%%%%%%%%%%%%%%%%%%%
\section{Result and discussion}
\label{sec:result}

The PVT runs are shown in Figure \ref{fig:PVT2by2}

\begin{figure}[!htb]
	\centerline{\includegraphics[width=0.88\textwidth]{fig_halfLives.pdf}}
	\caption{The red lines show the results of the fits over the runs that were used.
		 The additional half-lives shown are excluded to reasons discussed previously. 
		}
	\label{fig:PVT2by2}
\end{figure}

The results for the PVT runs are shown in  in table \ref{tab:PVTTable} for the PVT runs.
%\begin{center}
	\begin{table}[!hbt]
			\caption{The PVT runs}
			\begin{tabular}{l|r|r}
			Detector & Half-Life & Error \\ \hline
			Up & 11.0194 & 0.0090 \\
			Left & 10.9971 & 0.0090 \\
			Down & 10.9980 & 0.0085 \\
			Right & 10.9997 & 0.0095 \\ \hline
			Mean & 11.0010 & 0.0045
			\end{tabular}
			\label{tab:PVTTable}
	\end{table}
%\end{center}

The size of the systematic errors investigated are shown in table \ref{tab:SysTable} 

%\begin{center}
\begin{table}[!hbt]
	\caption{Systematics}
	\label{tab:err-budget}
		\begin{tabular}{l|r|r}
		Source & Correction [ms] & Uncertainty [ms] \\ \hline
		Dead-time & 0.00 & 0.24 \\
		Oscillator stability & 0.00 & 0.80 \\
		Lower PVT cut & 0.00 & 2.32 \\
		Lower gamma cut & 0.00 &  0.15\\
		Upper gamma cut  & 0.00 & 0.05 \\ 
		Relitive time cut & 1.64 & 1.64 \\
		Binning & -0.30 & 0.30 \\ \hline
		Total systematic & 1.34 & 2.98
		\end{tabular}
	\label{tab:SysTable}
\end{table}
%\end{center}
After the dead time correction, the value of the half-life from the CsI(Na) implant detector runs was found to be 11.0023 $\pm$ 0.0066 (stat) $\pm$ 0.0030 (syst) s.

%%%%%%%%%%%%%%%%%%%%%%%%%%%%%%%%%%%%%%%%%%%%%%%%%%%%%%%%%%%%%%%%%%%%%%%%%%%%
\section{Conclusion}
\label{sec:conclusion}

The half-life measured is most consistent with some previous measurements of a half-life of about 11 s. 
This can be seen in Figure \ref{fig:ideogramfinal}.
This measure disagrees with the most recent measurements.

\begin{figure}[!htb]
\centerline{\includegraphics[width=0.88\textwidth]{fig_ideogramAfter.png}}
\caption{A scatter plot of (a) previous values with this work added.
	 The labels correspond to: Mal~\cite{Mal62}, Gli~\cite{Gli63},
	Yul~\cite{Yul67}, Wil~\cite{Wil70}, Alb~\cite{Alb75}, Gen~\cite{Gen76},
	Min \cite{Min87}, Wan~\cite{Wan92} and Ito~\cite{Ito95}.}
\label{fig:ideogramfinal}
\end{figure}

Several things about the data set were discovered.
It was learned that there were contaminates coming from the fragmentation of the $^{12}$C in the plastic scinttilator.
The beta decaying products could be a cause of why the beta spectrum for the PVT detector looks so odd.
It was also discovered that the rate dependent gain effect in the CsI(Na) implant detector is significant.
Before doing this measurement, it was assumed that the low rates would make that effect negligable.
It was learned that after the coincidence, the amount of background left in the spectrum is negligable, and that we have no contaminates in the beam.


%%%%%%%%%%%%%%%%%%%%%%%%%%%%%%%%%%%%%%%%%%%%%%%%%%%%%%%%%%%%%%%%%%%%%%%%%%%%
%Bibliography

%%%%%%%%%%%%%%%%%%%%%%%%%%%%%%%%%%%%%%%%%%%%%%%%%%%%%%%%%%%%%%%%%%%%%%%%%%%%
%%%%%%%%%%%%%%%%%%%%%%%%%%%%%%%%%%%%%%%%%%%%%%%%%%%%%%%%%%%%%%%%%%%%%%%%%%%%
