\documentclass[main.tex]{subfiles}

\begin{document}
Precision measurements at low energy can compete with high energy measurements.

\section{New Physics Searches}
\section{Beta Decay}
\subsection{Types of Percision Measurements in Beta Decay}
Beta decay is cool and good.
\subsection{Matrix elements of Beta Decay}
When the beta decay rate is calculated, there are several terms.
There are phase space terms, which originate from the density of states.
These and the electromagnetic interactions will be discussed further into the thesis. 
The contribution to beta decay $M$ from the weak interaction, to first order, is shown in equation \ref{eq:decayrate} \cite{Gon19}

\begin{equation}
	M = \xi [1 + a \frac{\vec{p_{e}} \dot \vec{p_{\nu}}} {E_{e} E_{\nu}}  +  b \frac{m_{e}}{E_{e}} + \frac{\vec{<J>}}{J} \dot (A \frac{ \vec{p_{e}} }{E_{e}} + B \frac{\vec{p_{\nu}}}{E_{\nu}} + D \frac{\vec{p_{e}} \times \vec{p_{\nu}}}{E_{e} E_{\nu}})]
	\label{eq:decayrate}
\end{equation}
where $\xi$ is a constant that depends, to first order, on the Lee-Yang coupling constants, $\vec{p}$  is the momentum of either the electron $e$ or the neutrino $\nu$, and $E$ the energy of the same particle.
$<\vec{J}>$ is the average total angular momentum. 
The constants $a$, $b$, $A$, $B$, and $D$ can be written in terms of the coupling constants as well.
$a$ depends on the mixing of the Gamow-Teller and Fermi matrix elements.
A polarized nucleus is need to extract $a$. 
For an unpolarized beam where only the energy of the electron is measured, the momenta are averaged over and all terms disappear expect for $b$.
This term is known as the Fierz term, and a measurement of the Fierz term was the ultimate goal of this experiment. 

\section{Fierz Term}
The Fierz term, $b$, in equation \ref{eq:decayrate}, can be rewritten in terms of effective couplings.
This is shown in equation \ref{eq:bwrittenout}

\begin{equation}
	b =  \pm \sqrt{1 - \alpha^{2}{Z^{2}}}\frac{1}{1 + \rho}Re(\frac{C_{S} + C_{S}'}{C_{V} + C_{V}'} + |\rho|^{2}\frac{C_{T} + C_{T}'}{C_{A} + C_{A}'})
	\label{eq:bwrittenout}
\end{equation}

where $\rho$ is the ratio of the Gamow-Teller matrix element to Fermi matrix element, $\alpha$ is the QED fine structure constant \cite{Gon19}
The subscripts of $C$ indicate which object the coupling corresponds to. 
Here, $A$ means axial vector, $V$ stands for polar vector, $S$ stands for scalar, and $T$ stands for tensor. 
The $C$ coefficients correspond to parity conserving interactions, and the $C'$ coefficents correspond to parity non-conserving coefficents \cite{Lee56}
This means, that in a pure Fermi decay, the Fierz term is sensitive to any non-standard scalar term, while in a pure Gamow-Teller decay, the Fierz term is sensitive to any non-standard Tensor terms. 
The Fermi decays have been measured using super-allowed beta decays, where the spin change is $0^{+}$ to $0^{+}$. 
In order to get a good measurement of any tensor couplings, an allowed Gamow-Teller decay must be used. 
 

\end{document}
