\documentclass[../MaxHughesThesis.tex]{subfiles}

\begin{document}
A digital data acquisition (DAQ) system was used to record the data for this experiment.
To configure the DAQ, the detector signals were looked at using an oscilloscope in order to get the decay time.
The input parameters of the energy filter were tuned from the decay time of the signals.

\section{Data Acquisition}

All the signals for all the detectors were sent to 16 channel XIA (X-ray Instrumentation Associates) Pixie modules. 
Each channel had a 250 mega-sample per second digitizer. 
The incoming signal was first digitized, and then processed by the module. 
The modules were run with the NSCL firmware.
A list of the three modules and which channel processed the signals from which detector is seen in table \ref{tab:cablingscheme}.
%
\begin{table}[!hbt]
	\centering
	\caption{DAQ cabling Scheme.}
		\begin{tabular}{llr} \hline \hline 
		Module & Channel & Detector \\ \hline
		0 & 0 & PVT Implant \\
		0 & 1 & CsI(Na) Implant \\
		0 & 4 & Large CsI(Na) Up \\
		0 & 5 & Large CsI(Na) Left \\
		0 & 6 & Large CsI(Na) Down \\
		0 & 7 & Large CsI(Na) Right \\
		0 & 8 & Beam-on Signal \\
		0 & 9 & Beam-off Signal \\
		0 & 10 & 100 Hz Pulser \\
		0 & 11 & PVT Gain Monitoring Pulser \\
		1 & 0 & PPAC Up \\ 
		1 & 1 & PPAC Down \\ 
		1 & 2 & PPAC Left \\ 
		1 & 3 & PPAC Right \\ 
		1 & 8 & Beam-on Signal\\
		1 & 9 & Beam-off Signal\\
		1 & 10 & 100 Hz Pulser \\
		1 & 11 & PVT Gain Monitoring Pulser \\
		2 & 0 & PVT Implant (With Digitized Wave Forms) \\ \hline \hline
		\end{tabular}	
		\label{tab:cablingscheme}
\end{table}
%
For module 2 channel 0, the digitized wave-form was recorded over 400 ns.
The clock used for the digitizers was a stable EPSON SGR-8002JC programmable crystal oscillator.
The first thing the Pixie modules did was to digitize the wave forms of the signals.
The digital wave forms of the signals were only saved for the PVT implant detector.
The Pixie modules' software applied a trapezoidal energy filter to the digital wave-forms.
Each of the channels gave a time stamp and an energy calculated from a trapezoidal filter.

\subsection{Trapezoidal Filter Description}
In order to get timing and energy information for the digitized signals, a trapezoidal filter was applied.
After the parameters of the filter were set, the filter was taken on the digitized detector signals.
First, three sums of the digitized wave-forms were taken.
The first sum was over a time known as $t_{PEAKING}$.
Then, a second sum was taken directly after the first over a time known as $t_{GAP}$.
Then, a third sum was taken directly after the the second with over a period of $t_{PEAKING}$ again.
These three sums were associated with weights given as \cite{Tan03} 

\begin{equation}
	E = C_{0} \times S_{0} + C_{g} \times S_{g} + C_{1} \times S_{1} 
	\label{eq:ensum}
\end{equation} 
%
with $E$ being the energy of the signal, $S_{0}$ being the first sum over $t_{PEAKING}$ described above, $S_{g}$ being the sum over $t_{GAP}$, and $S_{1}$ being the second sum over $t_{PEAKING}$.
$C_{0}$ is given by 

\begin{equation}
	C_{0} = \frac{-(1 - e^{\frac{t_{SPL}}{\tau}})e^{\frac{t_{PEAKING}}{\tau}}}{1 - e^{\frac{t_{PEAKING}}{\tau}}}
	\label{eq:c0sum}
\end{equation}
%
where $t_{PEAKING}$ is the time described above, $t_{SPL}$ is the 8 ns sampling time, and $\tau$ an additional parameter.
$C_{g}$ is 

\begin{equation}
	C_{g} = 1 - e^{\frac{t_{SPL}}{\tau}}
	\label{eq:cgsum}
\end{equation}
%
Finally, $C_{1}$ is 

\begin{equation}
	C_{1} = \frac{1 - e^{\frac{t_{SPL}}{\tau}}}{1 - e^{\frac{t_{PEAKING}}{\tau}}}
	\label{eq:c1sum}
\end{equation}
%
This filter averaged the background before a pulse in $C_{0}$ over $t_{PEAKING}$.
It took a sample of the pulse, averaged over $t_{PEAKING}$ again in $C_{1}$.
Then, it subtracted off $C_{0}$ from $C_{1}$, after adjusting for the exponential decay of the pulse.
The time constant of that decay is the $\tau$ parameter.
When tuning the filter, the $\tau$ parameter had the largest effect.
For logic signals which do not decay, the $\tau$ parameter was very large.
For this experiment, the various filter parameters are summarized in table \ref{tab:pixieparams}.  
%
\begin{table}[!hbt]
	\centering
	\caption{Energy filter parameters.}
		\begin{tabular}{lrrr} \hline \hline 
			Detector & $\tau$ & $t_{PEAKING}$ & $t_{GAP}$ \\ \hline
			PVT implant & 60 ns & 208 ns & 128 ns \\
			All CsI (Na) detectors & 900 ns & 480 ns & 48 ns \\
			PPAC channels & 300000 ns & 1200 ns & 96 ns \\ 
			Si PIN & 50000 ns & 1744 ns & 96 ns \\
			Beam-on/beam-off signals & 1000000 ns & 400 ns & 96 ns \\
			Pulser & 5000000 ns & 400 ns & 96 ns \\
			Light pulser & 150 ns & 608 ns & 96 ns \\ \hline \hline
		\end{tabular}	
		\label{tab:pixieparams}
\end{table}
%
There was also a trigger filter in the Pixie system.
The expression in equation \ref{eq:ensum} is the same, but the parameters are shorter.
There was a threshold that the output of the trigger filter had to be above in order to be recorded.
These values are shown in table \ref{tab:trigfilter}.
The trigger filter was used to calculate the time stamps of the events.  
%
\begin{table}[!hbt]
	\centering
	\caption{Trigger filter parameters.}
		\begin{tabular}{lrrr} \hline \hline
			Detector & $t_{PEAKING}$ & $t_{GAP}$ & Threshold\\ \hline
			PVT implant & 104 ns & 104 ns & 30 \\ 
			All CsI (Na) Detectors & 200 ns & 72 ns & 20 \\
			PPAC Channels & 904 ns & 104 ns & 200 \\ 
			Si PIN & 400 ns & 80 ns & 100 \\
			Beam on/beam-off signals & 104 ns & 40 ns & 20 \\
			Pulser & 104 ns & 40 ns & 20 \\
			Light pulser & 104 ns & 72 ns & 35 \\ \hline \hline 
		\end{tabular}	
		\label{tab:trigfilter}
\end{table}
%

\subsection{Data Acquisition Software}
In order to set the parameters of the various filters, a program called NSCOPE was used \cite{DAQ17}.
The program was used to set the rise time, the gap, and the threshold for both the energy filter and the trigger filter.
These parameters were set differently for each detector type.
The CsI(Na) detectors (the implant and the 4 gamma detectors) shared the rise time and gap time, while other detectors had different parameters.
Sample spectra were taken with NSCOPE, and the parameters saved to a file.

This file was loaded with a program called ReadOut, which ran the data taking.
The ReadOut program used a ringbuffer in order to record the output of the data.
The program recorded data as an .evt binary file if the record option was selected.

The ringbuffer was also fed to a program called scalers.
This gave the rates of the events coming into each channel.
It gave an input rate to each channel, the recorded rate, and the total over each run.
This was to check if every channel in each module was still recording data.

For online analysis, a program called SpecTcl was used.
It took the .evt files and processed them into histograms.
These histograms could be put in coincidence.
The implant detector energy and rate were displayed.
The energy of the implant vs time was also displayed.
Additionally, the implant energy vs time since last beam-on was plotted as a 2-D spectra.
The decay curves were also plotted.
These were built by histogramming the time stamp in the detector relative the time since the last beam-on.
These histograms were used as diagnostics. 
The program allowed for building spectra with coincidence conditions. 
%Sample spectcl plot?

\section{Data Taking and Processing}
To take data, the ReadOutGUI program was activated. 
Data was taken in one hour intervals called runs.
If a channel stopped counting, the run was ended before one hour was up. 
The events built by Readout had a time interval of 8000 ns.

Each run generated one or more .evt files. 
The resulting .evt files of ReadOut were processed using a program called DDAS dumper.
This converted the .evt files to ROOT files.
One .evt was dumped to one .root file. 
These ROOT files contained a TTree that had the energy and time stamps of each detector.
The information was saved event by event and sorted by time. 
It was formatted in a C++ class called DDASEvent. 
This event class contained data for each channel from the data acquisition.
The energy was given out in ADC units and was not calibrated.
These units had a range from 0 to 32767.
The time was given in counts of DAQ clock.
The digitized wave forms were also saved in this event. 

From the ROOT files produced by the ddasdumper, another ROOT file was built.
All the ROOT files from one run were built into one new ROOT file. 
These files were processed with a modified version of a program called scan.
The original version was written by Stan Paulauskas.
The TTree was looped over event by event.
For each event, each channel was looped over. 
For the first event and first channel, the time integer was recorded. 
This first time was subtracted from all subsequent time stamps.
The time differences were multiplied by 8 ns to turn them into time in seconds. 
Then, the various energies and times for each detector were built into histograms for each channel. 
During looping through the events, the energy and times for the implant detectors, four gamma detectors, beam-on signal, and beam-off signals were saved to a variable.
This information was used to build new events for a new TTree.

Each event started with a non-zero energy reading in any of the five detectors.
Then, a 400 ns long time gate opened, during which all energies and times were saved into the event.
The time signatures of the last beam-on and beam-off signals were also saved into each event.
The beam-on and beam-off signals were logic Nuclear Instrumentation Module (NIM) signals.

%\begin{table}[!hbt]
%	\centering
%	\caption{Summary of programs used.}
%		\begin{tabularx}{\textwidth}{RLLLL}
%			Program & Inputs & Ouputs & Author & Purpose \\ \hline
%			NSCOPE \cite{DAQ17} & Pixie output & Energy histograms & Ron Fox & Tuning filter parameters \\
%			ReadOut \cite{DAQ17} & Pixie output & Events in the binary file & Ron Fox & Recording data to disk \\ 
%			DDAS Dumper \cite{DAQ17} & Events in the binary file & ROOT TTree & Ron Fox & Processing events to ROOT format \\
%			scan & ROOT Tree & Histograms and simpler ROOT TTree & Stan Paulauskus & Further processing of ROOT format \\
%			ROOT \cite{Ant09} & Root files & Histograms & CERN & For data analysis 
%		\end{tabularx}	
%		\label{tab:programs}
%\end{table}

%A summary of all the programs used is show in table \ref{tab:programs}.

\end{document}
