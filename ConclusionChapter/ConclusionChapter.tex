\documentclass[../MaxHughesThesis.tex]{subfiles}

\begin{document}
There is still a mismatch between the simulated data and the experimental data.
This is revealed in that changing the lower beta cut changes the fit value of $b_{WM}$ and $b_{GT}$.
How the simulation deals with the absorption of gamma rays and the detector geometry will have to be investigated.
However, most of the systematic effects have been calculated.

\section{Discussion of Results}

The statistical uncertainty of $b_{GT}$ is 0.0042.
This is with $7 \times 10^{6}$ events that pass the gamma coincidence.
%If the lower beta cut uncertainty is excluded, the total systematic uncertainty is 0.0043.
If the lower beta cut uncertainty is excluded, the total systematic uncertainty is 0.0076.
This is larger than the statistical uncertainty.
The largest systematic effect is the uncertainty on the size of the beta-gamma sum spectrum.
This is a conservative estimate of the uncertainty and could change.
The next largest systematic effect is the uncertainty on the $b_{WM}$.
Increasing the number of counts will not decrease the uncertainty of $b_{GT}$ below 0.0034.

%In order to compare this to high energy techniques, equation \ref{eq:bgtpropor} is used \cite{Gon19}.
%If the lower beta cut effect is resolved and the systematic effects described above accurate, then the total uncertainty on $b_{GT}$ is 0.0078. 
%This corresponds to an uncertainty on $Re(\epsilon_{t})$ of 0.0013.
%This is about twice the uncertainty from the neutral current channel at the LHC. 
%Running the experiment with more statistics would help reach that uncertainty.
%Then, the statistical uncertainty would be negligible, and the systematic uncertainty would be the limiting factor. 

\section{Further Refinements to the Technique}
This measurement is already systematics limited, so running more events would not improve sensitivity.
However, other steps could be taken to improve the technique.

The largest systematic effect right now is the sensitivity to the lower beta cut. 
A likely way to solve this issue is to verify the shape of the fitting functions. 
A method would be to compare the output of GEANT4 to that of another simulation code.
EGSnrc is being used by the collaborators from Wittenburg University. 
These results will be added to the systematic uncertainty.

Ideally, the lowest lower beta cut should be used for the fit.
This both increases the statistics in the spectrum and decreases the uncertainty due to the uncertainty in $b_{WM}$.
Going below $200$ keV for this measurement is not possible due to the non-linearities in the detector.
If the issue with the shape of the PVT detector spectrum could be solved, the spectrum could be measured down to smaller energies, since plastic detectors are linear down to about 125 keV \cite{Kno10}. 
There would be an increased sensitivity to the Fierz term due to the $1/E$ functional dependence of that term.

In order to increase the efficiency for the 1.6 MeV gamma ray, the implant detector could be more completely encased with gamma detectors.
If these gamma detectors had a higher resolution, that would mean that the signal in the implant would be much cleaner.
If that were coupled with a plastic detector, then the beta spectrum of the implant detector would be much cleaner.
This would hopefully eliminate the amount of gamma-beta sum spectrum and reduce the amount of bremsstrahlung.

If these systematic effects are reduced, the next largest one has to do with the uncertainty on the measurement of $b_{WM}$.
To decrease the uncertainty of $b_{WM}$,  a new and better measurement of the width of the isobaric analogue state in $^{20}$Ne would have to be measured.
This is the parameter $\Gamma_{M1}$ in equation \ref{eq:bwmcal}.
This is the parameter that dominates the uncertainty in  the calculation of $b_{WM}$ \cite{Min11}.
This would require a different type of experiment.

\section{Conclusion}
This calorimetric technique could be used for other allowed Gamow-Teller transitions.
These other transitions would need to have a more precisely know $b_{WM}$ to a more precise value.
Those with higher maximum energies would need an accelerator powerful enough to implant the beam deep enough to absorb all the electron energy.
A nucleus with a lower maximum energy, such as $^{6}$He, would be more sensitive to the 1/$E$ dependence of the Fierz term. 
This technique is a promising method for precision measurements in nuclear beta decay.

\end{document}
