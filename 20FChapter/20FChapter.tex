\documentclass[main.tex]{subfiles}

\begin{document}

For a potential beta decay shape measurement, several criteria for a nucleus are needed.
The first is that the nucleus and the decay mode must be available in sufficient quantities that a proper decay spectrum shape can be measured.
This means that the nucleus needs to be relatively close to stability.
The decay mode needs to be clean.
If there are several competing decay modes of similar stength, the shape of the beta decay spectrum gets complicated.
The same thing happens if there are several gamma rays in the decay. 
Having one gamma ray is useful, as a coincidence measurement can be used to exclude much of any background spectra in the measurement.
In order to get a useful theoretical interpretation of the measurement of the Fierz term, a purely Fermi or Gamow-Teller transition is needed.
The superallowed decays are purely Fermi, so the limit on any potential scalar couplings is strong.
An allowed Gamow-Teller transition gives sensitivity to any potential couplings. 
A nucleus that fulfils these criteria is $^{20}$F. 
 
\section{$^{20}$F Decay Characteristics}
The decay scheme is given in figure \ref{fig:DecayScheme}.

\begin{figure}[!htb]
	\centerline{\includegraphics[width=0.78\textwidth]{20FDecayScheme.png}}
	\caption{The decay scheme of $^{20}$F.}
	\label{fig:DecayScheme}
\end{figure}

As seen in the figure, $^{20}$F decays 99.999\% of the time to the first excited state of $^{20}$Ne.
This decay is very clean, as there are very few contaminants from other decay branches. 
The $2^{+}$ state seen in the decay scheme is not the isobaric analogue state to the ground state of $^{20}$F.
That state is much higher in energy.
The beta decay therefor has a isospin change of 1.
This means that the allowed Fermi matrix element is zero, and the decay is an allowed Gamow-Teller transition. 
The forbidden transitions contribute <INSERT THEORY HERE> 
The half-life of $^{20}$F is about 11 seconds. 
This means waiting for the $^{20}$F to decay will take less than a minute to get good statistics.

The 1.6 MeV gamma ray in the transition allows for a coincidence measurement to take place.
One gamma ray does not distort the spectrum enough to cause any issues.
$^{20}$F is realtively close to stability. 
It can be made in sufficient qualities to get a good beta quality beta spectrum.
Because of these characteristics, this measurement was not the first to measure $^{20}$F.

\section{Previous Measurments of $^{20}$F}



In order to get at the Fierz term, the beta decay spectrum shape must be described percisely.
The next chapter deals with the corrections to the beta decay spectrum.  

\end{document}
